
\documentclass{CSICC2020}





\title{
یک الگوریتم جستجوی فاخته بهبود یافته بر اساس DV-Hop برای مکان در wsn
}
\date{}

\author{نویسندگان: Xianfeng Ou, Meng Wu, Siyuan Chen , Wujing Li , Guoyun Zhang }




\begin{document}
\maketitle
\begin{abstract}
الگوریتم جستجوی کوکو بهبود یافته ای بر اساس الگوریتم DV-HOP برای موقعیت یابی در شبکه های حسگر بی سیم استفاده می شود . الگوریتم جستجوی کوکو از رفتار جوجه کوکو الهام گرفته شده است و یک الگوریتم بهینه سازی فراابتکاری است که برای حل مسائل بهینه سازی مختلف، از جمله مسائل موقعیت یابی در شبکه های حسگر بی سیم استفاده می شود. الگوریتم DV-Hop نیز یک الگوریتم معروف موقعیت یابی در شبکه های حسگر بی سیم است که با استفاده از فاصله های بین گره ها، به تخمین موقعیت گره ها می پردازد. در این مقاله، یک الگوریتم بهبود یافته جستجوی کوکو با استفاده از منطق فازی و استراتژی گاس-کوشی پیشنهاد شده است که الگوریتم فراابتکاری را با روش سنتی ادغام می کند. برای تنظیم تطبیقی تنظیمات پارامترها، ما یک منطق فازی بر اساس تنوع جمعیت پیشنهاد می دهیم .

 \end{abstract}
\begin{keywords}
حسگر ، بی سیم ، گاوس کوشی ، موقعیت یابی ، گره ، الگوریتم ، فاخته ، WSN
\end{keywords}

\section{مقدمه}
امروزه با عمیق تر شدن عصر اطلاعات دیجیتال،شبکه حسگر بی سیم (WSN) به یک زمینه تحقیقاتی مهم با طیف گسترده ای از کاربردها، مانند نظارت بر وضعیت، ردیابی هدف و نظارت بر دستگاه های پزشکی و غیره تبدیل شده است.در شبکه های حسگر بی سیم، تعداد بسیار زیادی از گره های حسگر در زمینه های مختلفی مانند نظارت بر شرایط، ردیابی هدف و مانیتورینگ دستگا ههای پزشکی و ... مستقر می شوند. این گره ها با یکدیگر ارتباط برقرار کرده و با این کار برای کنترل یا نظارت بر پارامترهای محیطی مانند دما و نور در زمان واقعی، تلاش می کنند. با این حال، دقت محدود قرارگیری این رویکرد، یک مشکل ناگزیر است. بنابراین، یک الگوریتم جستجوی فاخته بهبود یافته با منطق فازی و استراتژی گاوس-کوشی (ICS-FG) برای بهبود دقت موقعیت یابی با ساختار یک مدل DV-HOP پیشنهاد شده است. اول از همه، درجه ازدحام جمعیت فضای جستجو را می توان برای نشان دادن قابلیت جستجوی الگوریتم در طول تکرار استفاده کرد. از سوی دیگر، تنظیم پارامترها این پتانسیل را دارد که بر قابلیت جستجوی الگوریتم تا حدی تأثیر بگذارد و آن را محدود کند. در نتیجه، یک منطق فازی بر اساس تنوع جمعیت برای کنترل به روز رسانی پارامترها پیشنهاد شده است. سپس، استراتژی پیاده روی تصادفی ترجیحی در بهره برداری از الگوریتم برای مطابقت با پرواز   Lévy  از نظر قابلیت جستجو کافی نیست و عدم تعادل بین اکتشاف و بهره برداری را نشان می دهد. بنابراین، رویکرد گاوسی-کوشی توسعه یافته است، که یک استراتژی به روز رسانی جمعیت مبتنی بر توزیع های گاوسی و کوشی است. نتایج آزمایشی نشان می دهد که روش ICS-FG  پیشنهادی در نهایت، در حل تابع بنچمارک و مسئله موقعیت یابی گره های نامعلوم WSN برتری و رقابت پذیری دارد.
\section{الگوریتم ICS-FG}
در رویکرد پیشنهادی ICS-FG منطق فازی بر اساس تنوع جمعیت طراحی شده است تا به صورت پویا احتمال کشف a , b و فاکتور اندازه گام a را بروز کند و از این طریق تأثیر آن بر قابلیت کاوش الگوریتم جستجوی کوکو (CS) را کاهش دهد. در ادامه، یک روش گاؤس-کوشی بر مبنای تعداد تکرارهای کنونی طراحی شده است تا قابلیت جستجوی الگوریتم را در بخش بهره برداری بهبود دهد و تعادلی بین قابلیت کاوش و بهره برداری داشته باشد. در روش ICS-FG ، با توجه به نتایج مقایسه بین فاکتور احتمال 𝑃𝑎 و عدد تصادفی 𝑟𝑎𝑛𝑑() ∈ (0, 1)، بعد از به روزرسانی جمعیت با استفاده از پرواز لووی، افراد خاصی را به روز می کند که با استفاده از روش گاؤس-کوشی انجام می شود. با این حال، این مقاله فرض می کند که جمعیت دو بار در هر تکرار به روز می شود تا درک و اجرا آسان تر شود. پس از پیاده سازی روش ICS-FG ، آن را بر روی مسئله ی موقعیت یابی گره های شبکه حسگری بر اساس الگوریتم DV-Hop  اعمال کردیم. 
به طور ساده، رویکرد ICS-FG برای بهینه سازی موقعیت یابی در شبکه های حسگر بیسیم استفاده میشود. این رویکرد بر اساس دو مفهوم اصلی، یعنی خوشه بندی و گرانش فازی، عمل میکند. مرحله اول این رویکرد، خوشه بندی است. به طور تصادفی، حسگرها را به چند خوشه مختلف تقسیم میکنند. هر خوشه حاوی تعدادی حسگر است . سپس، در مرحله دوم، موقعیت بهینه حسگرها در هر خوشه محاسبه میشود. این موقعیت ها بر اساس فاصله بین حسگرها و میزان اطمینان )مثلاً دقت اندازه گیری( تعیین میشود. این مرحله به صورت تکراری انجام میشود تا موقعیت بهینه حسگرها در خوشه ها تعیین شود . 
در مرحله بعدی، خوشه ها با استفاده از گرانش فازی بررسی میشوند. گرانش فازی، تاثیر حسگرها بر روی خوشه ها را نشان میدهد. خوشه های بهینه با توجه به معیارهای مشخص شده انتخاب میشوند . از آنجا که مراحل 2 و 3 تکراری هستند، روند بهینه سازی ادامه پیدا میکند تا موقعیت بهینه حسگرها و خوشه ها به دست آید . با استفاده از رویکرد ICS-FG ، میتوانید موقعیت یابی در شبکه های حسگر بیسیم را بهبود بخشید و با استفاده از معیارهای مورد نظر خود، موقعیت بهینه حسگرها و خوشه بندی را تعیین کنید. اما بهتر است برای مطالعه دقیق تر و پیاده سازی، به منابع تخصصی در این زمینه مراجعه کنید.

\section{مزایا الگوریتم ICS-FG }

دقت و سرعت بالا: ICS-FG موقعیت گره‌ها را به سرعت و با دقت قابل‌توجهی به دست می‌آورد، که به بهبود عملکرد کلی سیستم WSN کمک می‌کند.

کاهش اضطراب از دست دادن اطلاعات: با اطمینان از دقت موقعیت‌یابی، ICS-FG به طور قابل‌توجهی احتمال از دست رفتن اطلاعات را کاهش می‌دهد، که برای کاربردهایی که به داده‌های حسگر قابل اعتماد نیاز دارند، مانند مراقبت‌های بهداشتی و اتوماسیون صنعتی، بسیار مهم است.

مناسب برای اینترنت اشیاء (IoT): ICS-FG به طور خاص برای WSN‌های بزرگ و پیچیده که در اینترنت اشیاء (IoT) استفاده می‌شوند، طراحی شده است. این نرم‌افزار می‌تواند به طور کارآمد با مقادیر زیادی از داده‌های حسگر در زمان واقعی مقابله کند.

کارآمدتر از روش‌های دیگر: آزمایشات نشان داده‌اند که ICS-FG از سایر روش‌های لوکالیزه کردن گره‌ها در WSN دقیق‌تر و کارآمدتر است.



\section{لوکالیزه کردن}
نرم افزار مقدار خطای کمینه لوکالیزه کردن گره ها در WSN را بازمی گرداند. کمینه خطای لوکالیزه کردن به معنای این است که اطلاعات موقعیت هر گره در منطقه پوشش WSN به سرعت و با دقت قاب لتوجهی به دست می آید که به کاهش اضطراب ناشی از از دست دادن اطلاعات کمک می کند. در حال حاضر، با شیوع مفهوم اینترنت اشیاء، شبکه های حسگر بی سیم نیز در بسیاری از حوزه ها به کار گرفته می شوند و تکنولوژی لوکالیزه گره ها بسیار حائز اهمیت است. برای توضیح قابلیت اعمال و برتری روش ICS-FG ، نتایج آزمایشی به لحاظ توابع تست مرجع و لوکالیزه گره های WSN تحلیل می شوند.

\section{مقایسه با الگوریتم های دیگر}

در این بخش، روش پیشنهادی ICS-FG با الگوریتم های CS ، DV-Hop ، CLPSO  ،  Breed-PSO ، IAGA  و  pcCS در آزمایش شبیه سازی مکان یابی گره های WSN مقایسه می شود. در این مقاله، یک مدل WSN  مربعی با طول ضلع 100 به صورت تصادفی تولید شده است که شامل 100 گره تصادفی است. 
برای اثبات پایداری الگوریتم پیشنهادیICS-FG ، آزمایشات شبیه سازی در شرایط مختلف انجام می شود، به عنوان مثال تغییر نسبت گره های لنگر یا تغییر شعاع ارتباطی گره ها در حالیکه سایر متغیرها ثابت باقیمانده اند.
در مدل WSN با 30 گره لنگر و 70 گره ناشناخته و شعاع ارتباطی 20 ، الگوریتم ICS-FG  به طور قابل توجهی بهبودی در خطای میانگین مکان یابی نسبت به الگوریتم های برجسته IAGA و CLPSO را به دست آورده است. الگوریتم ICS-FG در بعضی اوقات می تواند یک خطای مکان یابی بسیار عالی را که نسبت به ۵ متاهوریستیک دیگر قابل توجهاً پایین تر است به دست آورد. در شرایطی که مدل WSN  ثابت است و شعاع ارتباطی 20 است، این بخش با تغییر مستمر نسبت گره های انکور آزمایشات شبیه سازی انجام می دهد، مانند 10٪ ، 20٪ ، 30٪ و 40٪ . با تحلیل تصاویر بالا، می توان یافت که خطای مکان یابی میانگین7 الگوریتم به طور نسبی با افزایش نسبت گر ههای انکور کاهش می یابد. 
در صورتی که مدل WSN ثابت باشد، این بخش با تغییرات پویا در شعاع ارتباطی برای آزمایش عملکرد الگوریتم ICS-FG  در شعا عهای مختلف از جمله51 متر، 20 متر، 52 متر و 30 متر، آزمایشات انجام می دهد. این مشخص است که اگرچه با تغییر شعاع ارتباطی فاصله بین دیگر روش ها و الگوریتم ICS-FG کاهش یافته است، اما الگوریتم ICS-FG  همواره در مقام پیشگام بوده است.





\section{نتیجه گیری}
در این مقاله یک الگوریتم بهینه سازی جستجوی کوکو با منطق فازی و استراتژی گاوس-کوشی بهبود یافته (ICS-FG) پیشنهاد شده است. یک منطق فازی مبتنی بر تنوع جمعیت برای بهروزرسانی پارامترها و تعادل قابلیت جستجوی جهانی و استخراج محلی طراحی شده است. برای بهبود قابلیت جستجوی الگوریتم جستجوی کوکو در استخراج محلی، استراتژی گاوس-کوشی مبتنی بر توزیع گاوسی و کوشی ارائه شد. عملکرد روش پیشنهادی ICS-FG بر روی مجموعه توابع بنچمارک و محل یابی گره های شبکه حسگر بی سیم (WSN) تایید شد. رویکرد پیشنهادی برای اکثر توابع بنچمارک بهترین راه حل را در مقابل ۶ الگوریتم موجود معروف مانند IAGA ، pcCS و الگوریتم CLPSO ارائه می دهد. برای محل یابی گره های WSN ، الگوریتم ICS-FG پیشنهادی نسبت به الگوریتم هایی مانن IAGA  ، pcCS و  CLPSO عملکرد بهتری دارد. همه این نشان می دهد که الگوریتم پیشنهادی رقابتی و  برتری داشته و بهترین الگوریتم محل یابی گره های WSN است .


\section{منابع و مراجع}

 V.J. Hodge, S. O’Keefe, M. Weeks, A. Moulds, Wireless sensor networks for
condition monitoring in the railway industry: A survey, IEEE Trans. Intell. Transp.
Syst. 16 (3) (2015) 1088–1106.
 E. Masazade, A. Kose, A proportional time allocation algorithm to transmit binary
sensor decisions for target tracking in a wireless sensor network, IEEE Trans.
Signal Process. 66 (1) (2018) 86–100.
 M. Zhang, A. Raghunathan, N.K. Jha, MedMon: Securing medical devices through
wireless monitoring and anomaly detection, IEEE Trans. Biomed. Circuits Syst. 7
(6) (2013) 871–881.
 P.-C. Song, J.-S. Pan, S.-C. Chu, A parallel compact cuckoo search algorithm for
three-dimensional path planning, Appl. Soft Comput. 94 (2020) 106443.
 V.L. Huang, P.N. Suganthan, J.J. Liang, Comprehensive learning particle swarm
optimizer for solving multiobjective optimization problems, Int. J. Intell. Syst. 21
(2) (2006) 209–226.
 M. Campos, R.A. Krohling, Hierarchical bare bones particle swarm for solving
constrained optimization problems, in: 2013 IEEE Congress on Evolutionary
optimization of steam reformer with an improved optimization algorithm, Int. J.
Hydrogen Energy 38 (26) (2013) 11288–11302.
 A. Ouyang, Y. Lu, Y. Liu, M. Wu, X. Peng, An improved adaptive genetic algorithm
based on DV-Hop for locating nodes in wireless sensor networks, Neurocomputing
458 (2021) 500–510.


\end{document}


